\documentclass{pginz}

\usepackage{subcaption}
\usepackage{indentfirst}
\usepackage[T1]{fontenc}
\usepackage{csquotes}
\usepackage[
    backend=biber,
    style=numeric,
    sorting=ynt
]{biblatex}
\usepackage{float}
\usepackage{framed}
\usepackage{amsmath}
\usepackage{dirtytalk}

\hypersetup{
    colorlinks=true,
    linkcolor=blue,
    filecolor=magenta,
    urlcolor=cyan,
    pdftitle={Praca Magisterska},
}

\addbibresource{mag.bib}

\renewcommand{\arraystretch}{1.5}

\renewcommand{\ang}[1]{ang. \textit{#1}}
\newcommand{\todo}[1]{\textbf{Do zrobienia: } #1}
\newcommand{\mref}[1]{(\ref{#1})}
\DeclareMathOperator{\Loss}{\mathcal{L}}

\begin{document}

\includepdf[pages={1}]{title-pl.pdf}
% \includepdf[pages={1}]{inne/Oswiadczenie.pdf}
\setcounter{page}{3}

\todo{Abstrakt}

\tableofcontents
\addcontentsline{toc}{chapter}{Spis treści}

\newpage

\chapter{Wstęp}

Współczesne systemy informatyczne są złożone i często rozproszone pomiędzy różne niezależne komponenty, które działają asynchronicznie.
Zrozumienie ich zachowań i interakcji w tych systemach staje się dużym wyzwaniem, zwłaszcza podczas diagnozowania problemów.
Biorąc pod uwagę ograniczenia tradycyjnych narzędzi do rozwiązywania problemów, takich jak GDB, dla tak skomplikowanych środowisk,
inżynierowie często uciekają się do bardziej dostępnej, a jednak skutecznej techniki:
przechwytywania i zapisywania śladów aplikacji jako plików tekstowych, wzbogaconych o dodatkowe dane, takie jak znaczniki oznaczające czas.
Te pliki, powszechnie znane jako \say{logi} lub \say{pliki logów}, służą jako kluczowe zasoby do monitorowania, diagnozowania i rozumienia wewnętrznych mechanizmów współczesnych systemów informatycznych.

Pliki logów zazwyczaj przyjmują formę półstrukturalnych plików tekstowych, gdzie
każda nowa linia odpowiada nowemu zdarzeniu aplikacji.
Takie zdarzenie zazwyczaj również bezpośrednio wskazuje na konkretną lokalizację
w kodzie źródłowym aplikacji.
Dodatkowo, każda linia może zawierać dodatkowe informacje \mref{sec:log-structure}, takie jak znacznik
czasu, powagę informacji, lokalizację itp.
Problem jednak polega na tym, że nie ma ujednoliconego sposobu strukturyzowania
takich plików, więc w praktyce większość aplikacji ma zupełnie różne struktury
plików logów\cite{LogTypes}, m. in. w zależności od:

\begin{itemize}
  \item Języka programowania
  \item Bibliotek cyfrowych
  \item Systemu operacyjnego
  \item Dostępnych zasobów
\end{itemize}

Logi stanowią nieocenione źródło danych dla
większości nowoczesnych aplikacji, oferując głęboki wgląd w działanie aplikacji
bez generowania znaczących dodatkowych kosztów.
Ich wszechstronność sprawia, że znajdują zastosowanie w różnych rozwiązaniach
technologicznych, od zintegrowanych systemów wbudowanych po zaawansowane
rozproszone serwisy oraz usługi internetowe.
Ta uniwersalność i bogactwo zawartych informacji sprawiają, że logi są
przedmiotem intensywnych badań, zwłaszcza w kontekście ekstrakcji istotnych
danych, które mogą pomóc w usprawnieniu procesów diagnostycznych, monitoringu
wydajności oraz wykrywaniu potencjalnych problemów.

Analiza logów stanowi kluczowy element utrzymania i diagnozowania współczesnych
systemów informatycznych. Logi te służą jako cenne źródło informacji dla
programistów, administratorów systemów i zespołów wsparcia, umożliwiając im
zyskanie wglądu w zachowanie aplikacji i skuteczne identyfikowanie problemów,
bez potrzeby ponownego uruchamiania systemu.
Niestety, pomimo powszechności logów we współczesnym oprogramowaniu, brak
ustandaryzowanego formatu stwarza interesujące wyzwania w tej dziedzinie.

\todo{Co w tej pracy jest zaproponowane ?}

\section{Struktura logów}
\label{sec:log-structure}

Jak wspomniano wcześniej, pliki logów często przyjmują format semi-strukturalny.
Oznacza to, że w tym samym pliku, linijki logu powinny posiadać wspólną
strukturę.
Struktura ta opisuje to jak wygląda linijka oraz jakie informacje muszą w niej
się znajdywać (pomijając samą informację związaną z logiem).
W praktyce dane takie obejmują informacje takie jak obecny czas, istotność
wydarzenia (ang. Severity), nazwę modułu, etc.

Problem jednak polega na tym, że nie istnieje uniwersalny sposób na definiowanie
powyższej struktury.
Sprawia to więc, że zadanie polegające na jej automatycznym zrozumieniu przez
program komputerowy jest nietrywialne.
Zilustrować to można na dość prostym przykładzie, na którego potrzeby
sfabrykowano przykładową linijkę \ref{code:sample-line-1}.
Ma ona dość łatwą strukturę z dwoma metadanymi: datą oraz informacją o ważności.

\begin{figure}[ht]
\begin{verbatim}
01.01.2024 INFO Client connected
\end{verbatim}
\caption{Przykładowa linijka logu}
\label{code:sample-line-1}
\end{figure}

\todo{Dokończyć przykład}

\chapter{Przegląd literatury}

\todo{}

\chapter{Zaproponowane rozwiązanie}

\todo{}

\chapter{Eksperymenty}

\todo{}

\chapter{Podsumowanie}

\todo{}

\listoffigures
\addcontentsline{toc}{chapter}{Spis rysunków}
\listoftables
\addcontentsline{toc}{chapter}{Spis tabel}


\printbibliography[
    heading=bibintoc,
    title={Bibliografia}
]

\end{document}
