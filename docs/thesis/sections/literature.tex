\chapter{Przegląd literatury}

\section{Najlepsze dopasowanie}

Plik logu traktować można jak uporządkowany zbiór linijek $l$.
W przypadku problemu postawionego w pracy, jednym z podproblemów staje się
optymalne dopasowanie do siebie dwóch takich zbiorów, tak aby każda linijka z
jednego logu posiadała odpowiadającą linijkę z drugiego.
W pracy tej przyjęto, że dobrą miarą dopasowania dwóch linijek do siebie, jest
dystans czasowy pomiędzy wystąpieniem linijek.

Problem ten więc przedstawić można jako minimalne dopasowanie do siebie dwóch
niezależnych zbiorów: $A$ oraz $B$, gdzie pomiędzy każdą parą elementów z obu
zbiorów zdefiniowana jest funkcja odległości $d$. Dziedzina takiej funkcji
opisana jest w równaniu \ref{eq:dziedzina}, gdzie $c$ to dowolna wartość którą
można porównać z inną wartością z tej samej dziedziny \ref{eq:def-dystans}

\begin{equation}
  d: a \times b \rightarrow c, \quad a \in A, b \in B, c \in C \label{eq:dziedzina}
\end{equation}

\begin{equation}
  \forall_{c_1,c_2 \in C}\ c_1 \neq c_2 \implies c_1 > c_2 \lor c_2 > c_1 \label{eq:def-dystans}
\end{equation}

\todo{}

\subsection{Algorytm Węgierski}

Klasycznym rozwiązaniem takiego problemu, jest zastosowanie algorytmu
węgierskiego \cite{Hungarian}.
Algorytm ten w swojej oryginalnej postaci rozwiązuje problem dopasowania dla
każdego pracownika $i$ najbardziej optymalnej pracy $j$.
Optymalność pracy opisana jest z wykorzystaniem macierzy kwalifikacji $Q$, tak
że wartość $Q[i, j]$ opisuje kwalifikacje pracownika $i$ do pracy $j$.
Zadaniem jest znalezienie par pracowników oraz prac, tak aby uzyskać jak
największą sumę kwalifikacji.
Zakładając więc, że zbiór $S$ posiada takie pary, to problem można opisać
formalnie w równaniu \ref{eq:hungarian-goal}

\begin{equation}
  \min \sum_{(i', j') \in S} Q[i', j']\label{eq:hungarian-goal}
\end{equation}

Algorytm więgierski, umie znaleźć najlepsze takie dopasowanie z złożonością
obliczeniową opisaną jako $\mathcal{O}(n^4)$.
Ma on jednak wiele udoskonaleń opisanych w kolejnych pracach
\cite{Hungarian-o3-1}\cite{Hungarian-o3-2}, uzyskując złożoność obliczeniową
$\mathcal{O}(n^3)$.

Powoduje to, że algorytm węgierski i jego pochodne, stają się najbardziej
oczywistymi podejściami do rozwiązania omawianego problemu dopasowania.
W przypadku problemu dopasowania zbiorów $A$ oraz $B$, macierz $Q$ opisana jest
zgodnie z równanie \ref{eq:hungarian-logs}

\begin{equation}
  Q[i, j] = d(A_i, B_j)\label{eq:hungarian-logs}
\end{equation}

\todo{}

\subsection{Graf dwudzielny}

Łatwo zauważyć, że relację taką zapisać można w postaci grafu dwudzielnego $G(V, E)$.
W grafie tym, każda krawędź reprezentuje odległość pomiędzy pomiędzy elementami
z odpowiednio zbioru $A$ oraz $B$.
W takiej sytuacji, problemem staje znalezienie grafu $G'(V, E')$, takiego że \todo{}
