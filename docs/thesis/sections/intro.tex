\chapter{Wstęp}

Współczesne systemy informatyczne są złożone i często rozproszone pomiędzy różne niezależne komponenty, które działają asynchronicznie.
Zrozumienie ich zachowań i interakcji w tych systemach staje się dużym wyzwaniem, zwłaszcza podczas diagnozowania problemów.
Biorąc pod uwagę ograniczenia tradycyjnych narzędzi do rozwiązywania problemów, takich jak GDB, dla tak skomplikowanych środowisk,
inżynierowie często uciekają się do bardziej dostępnej, a jednak skutecznej techniki:
przechwytywania i zapisywania śladów aplikacji jako plików tekstowych, wzbogaconych o dodatkowe dane, takie jak znaczniki oznaczające czas.
Te pliki, powszechnie znane jako \say{logi} lub \say{pliki logów}, służą jako kluczowe zasoby do monitorowania, diagnozowania i rozumienia wewnętrznych mechanizmów współczesnych systemów informatycznych.

Pliki logów zazwyczaj przyjmują formę półstrukturalnych plików tekstowych, gdzie
każda nowa linia odpowiada nowemu zdarzeniu aplikacji.
Takie zdarzenie zazwyczaj również bezpośrednio wskazuje na konkretną lokalizację
w kodzie źródłowym aplikacji.
Dodatkowo, każda linia może zawierać dodatkowe informacje \mref{sec:log-structure}, takie jak znacznik
czasu, powagę informacji, lokalizację itp.
Problem jednak polega na tym, że nie ma ujednoliconego sposobu strukturyzowania
takich plików, więc w praktyce większość aplikacji ma zupełnie różne struktury
plików logów\cite{LogTypes}, m. in. w zależności od:

\begin{itemize}
  \item Języka programowania
  \item Bibliotek cyfrowych
  \item Systemu operacyjnego
  \item Dostępnych zasobów
\end{itemize}

Logi stanowią nieocenione źródło danych dla
większości nowoczesnych aplikacji, oferując głęboki wgląd w działanie aplikacji
bez generowania znaczących dodatkowych kosztów.
Ich wszechstronność sprawia, że znajdują zastosowanie w różnych rozwiązaniach
technologicznych, od zintegrowanych systemów wbudowanych po zaawansowane
rozproszone serwisy oraz usługi internetowe.
Ta uniwersalność i bogactwo zawartych informacji sprawiają, że logi są
przedmiotem intensywnych badań, zwłaszcza w kontekście ekstrakcji istotnych
danych, które mogą pomóc w usprawnieniu procesów diagnostycznych, monitoringu
wydajności oraz wykrywaniu potencjalnych problemów.

Analiza logów stanowi kluczowy element utrzymania i diagnozowania współczesnych
systemów informatycznych. Logi te służą jako cenne źródło informacji dla
programistów, administratorów systemów i zespołów wsparcia, umożliwiając im
zyskanie wglądu w zachowanie aplikacji i skuteczne identyfikowanie problemów,
bez potrzeby ponownego uruchamiania systemu.
Niestety, pomimo powszechności logów we współczesnym oprogramowaniu, brak
ustandaryzowanego formatu stwarza interesujące wyzwania w tej dziedzinie.

W tej pracy zaproponowano przykładowy system, próbujący rozwiązać część
opisanych powyżej problemów, jednocześnie minimalizując ilość założeń
początkowych, w celu uzyskania jak najlepszej uniwersalnośći.
System, jako swoje wejście przyjmuje dwa pliki wejściowe:

\begin{itemize}
  \item \textit{Baseline} - Plik ten powinien posiadać logi z przebiegu aplikacji, w
    momencie gdy działała ona w sposób prawidłowy.
    Preferowane jest aby okres zbierania logów z systemu dla tego pliku, był
    podobny bądź lekko dłuższy od pliku \textit{checked}.
  \item \textit{Checked} - Ten plik zawierać powinien logi z przebiegu
    aplikacji, która zachowała się w nieporządany bądź niespodziewany sposób.
    Każda linijka w tym logu, w wyniku działania systemu, będzie posiadała
    odpowiadającą sobie linijkę w pliku \textit{baseline} (o ile zostanie
    znaleziona) oraz heurystykę jak istotna jest dana linijka.
\end{itemize}

Dodatkowo użytkownik ma możliwość dostrojenia dodatkowych parametrów, w celu
dalszego usprawnienia skuteczności systemu.
Umożliwia to wykorzystanie wiedzy domenowej, kiedy użytkownik ma taką możliwość.

Postawionym zadaniem systemu jest \textbf{asysta} użytkownika, tak aby
znalezienie istotnych infomacji była ułatwiona w stosunku do innych, bardziej
manualnych metod.

\todo{TEZA}

\section{Struktura logów}
\label{sec:log-structure}

Jak wspomniano wcześniej, pliki logów często przyjmują format semi-strukturalny.
Oznacza to, że w tym samym pliku, linijki logu powinny posiadać wspólną
strukturę.
Struktura ta opisuje to jak wygląda linijka oraz jakie informacje muszą w niej
się znajdywać (pomijając samą informację związaną z logiem).
W praktyce dane takie obejmują informacje takie jak obecny czas, istotność
wydarzenia (ang. Severity), nazwę modułu, etc.

Problem jednak polega na tym, że nie istnieje uniwersalny sposób na definiowanie
powyższej struktury.
Sprawia to więc, że zadanie polegające na jej automatycznym zrozumieniu przez
program komputerowy jest nietrywialne.
Zilustrować to można na dość prostym przykładzie, na którego potrzeby
sfabrykowano przykładową linijkę \ref{code:sample-line-1}, oznaczoną dalej jako
$l'$ \ref{math:sample-structure1}.
Ma ona dość łatwą strukturę z dwoma metadanymi: datą oraz informacją o ważności.

\begin{figure}[ht]
\begin{verbatim}
01.01.2024 INFO Client connected
\end{verbatim}
\caption{Przykładowa linijka logu}
\label{code:sample-line-1}
\end{figure}

Warto w tym miejscu zastanowić się, jaki wzór (ang. pattern) algorytm powinien wykryć w podanym
(\ref{code:sample-line-1}) przykładzie.
Dość prostym wzorem mógłby być: \texttt{<timestamp> <severity> <*> connected},
gdzie nazwy wokół trójkątynych nawiasów reprezentują parametry dla danej linijki
logu.
Oznacza to więc, że dla dowolnej linijki logu $l$, parser $\mathbb{P}$ powinien umieć
znaleźć w niej wzór $p = \mathbb{P}(l)$, przyjmujący jako argument parametry $x$, z
wykorzystaniem których odtworzyć można daną linijkę logu $l = p(x)$.
Operacje te, wykorzystując podany przykład, opisane zostały poniżej:
\ref{math:sample-structure1}, \ref{math:sample-structure2} i \ref{math:sample-structure3}

\begin{gather}
  l' = \texttt{'01.01.2024 INFO Client connected'}
  \label{math:sample-structure1} \\
  p', x' = \mathbb{P}(l') = \texttt{'<timestamp> <severity> <*> connected'},
  [\texttt{'01.01.2024'}, \texttt{'INFO'}, \texttt{'Client'}]
  \label{math:sample-structure2} \\
  l' = p'(x')
  \label{math:sample-structure3}
\end{gather}

Łatwo wyobrazić sobie linijkę logu $l^*$, która przekazuje identyczną informację
co linijka $l'$, jednak ma zupełnie inną strukturę.
Dla przykładu \ref{code:sample-line-1}, wymyślić można następujące logicznie
identyczne linijki \ref{code:sample-line-2}, różniące się tylko metadanymi,
notacją lub kolejnością.

\begin{figure}[ht]
\begin{verbatim}
01.01.2024 INFO Client connected
01/01/24 Information [thread 0] - Client connected
Info - 01 January 2024 - Client connected
\end{verbatim}
\caption{Przykładowa linijka logu - inne warianty}
\label{code:sample-line-2}
\end{figure}

\todo{Dokończyć przykład}
